\usepackage{ifxetex}
\ifxetex
	\usepackage{fontspec}
\else
	\usepackage[utf8]{inputenc}
\fi

\usepackage[UKenglish]{babel}
\usepackage[UKenglish]{datetime}
\usepackage[T1]{fontenc}
\usepackage{geometry}
\usepackage{moresize}
\usepackage{tocloft}
\usepackage{adjustbox}
\usepackage{graphicx}
\usepackage{xspace}
\usepackage{varwidth}
\usepackage{svg}
\usepackage{ifthen}
\usepackage{tikz}
\usetikzlibrary{calc,math,arrows,arrows.meta,
	decorations.pathreplacing,shapes.geometric,patterns,calligraphy}
\usepackage{pgfplots}
\pgfplotsset{compat=1.17}
\usepackage{pgffor}
\usepackage{pgfmath}
\usepackage[super]{nth}
\usepackage{romannum}
\AtBeginDocument{\pagenumbering{arabic}}
\usepackage{float}
\usepackage[font=footnotesize,justification=centering,margin=1cm]{caption}
\usepackage{makecell}
\usepackage{tabularx}
\usepackage{multirow}
%\renewcommand\tabularxcolumn[1]{m{#1}}
\usepackage{array}
\newcolumntype{L}{l@{}}
\newcolumntype{R}{r@{}}
\usepackage{amsmath,mleftright,amsthm,amsfonts,amssymb,amscd,nccmath}
\usepackage{centernot}
\usepackage{relsize}
\usepackage{physics}
\usepackage{polynom}
\usepackage{xparse}
\usepackage{fancyhdr}
\usepackage{titlesec}
\usepackage{accents}
\usepackage{mathtools}
\usepackage[scaled=1.15]{urwchancal}
\usepackage{bbding}
\usepackage{xfrac}
\usepackage{nicefrac}
\let\ifrac\nicefrac
\usepackage{etoolbox}
\usepackage[artemisia]{textgreek}
\usepackage{siunitx}
\usepackage{parskip}
\usepackage{multicol}
\usepackage{enumerate}
\usepackage{enumitem}
\usepackage{moreenum}
\usepackage{mathrsfs}
%\usepackage{lmodern}
\usepackage{slantsc}
\usepackage{bold-extra}
\usepackage{mfirstuc}
\usepackage{suffix}
\usepackage{contour}
\usepackage{ulem}
\usepackage{color,soul}

\def\noop{}

%% Alias some `soul` macros
\let\emph\relax
\DeclareTextFontCommand{\emph}{\itshape}
\let\strike\st
\let\under\ul
\undef\st
\undef\ul
\definecolor{lightblue}{rgb}{.90,.95,1}
\definecolor{lightgreen}{rgb}{.90,1,.95}
\sethlcolor{lightgreen}

%% Nice diagrams.
\usepackage[all,cmtip]{xy}

%% Vector notation.
\usepackage{harpoon}
\renewcommand{\vec}{\overrightharp}

%% Good underline (white background only).
\renewcommand{\ULdepth}{1.8pt}
\contourlength{0.8pt}
\newcommand{\ul}[1]{%
  \uline{\phantom{#1}}%
  \llap{\contour{white}{#1}}%
}

%% Cal fonts
\DeclareMathAlphabet{\pazocal}{OMS}{zplm}{m}{n}
\newcommand\mcal  [1]{\pazocal{#1}}
\newcommand\mcall [1]{\mathcal{#1}}
\newcommand\mcalll[1]{\mathscr{#1}}

%% Nice empty-set.
\newcommand\oldemptyset\emptyset
\renewcommand\emptyset{\mathlarger{\mathlarger\varnothing}}

%% Footnote style.
\renewcommand{\thefootnote}{[\arabic{footnote}]}

%% Header layout.
\ifdefined\noheader
	\noop
\else
	\fancyhf{}
	\headheight 14pt
	\fancyhead[RO]{\papertitle}
	\fancyhead[LE]{\paperauthor}
	\fancyhead[RE,LO]{\thepage}
	\pagestyle{fancy}
\fi

\makeatletter
\let\papertitle\@title
\let\paperauthor\@author
\makeatother


%% Theorems, definitions, remarks, lemmas, corollaries, &c.
\theoremstyle{plain}
\newtheorem{theorem}{Theorem}[section]
\newtheorem*{theorem*}{Theorem}
\newtheorem{corollary}{Corollary}[theorem]
\newtheorem{lemma}[theorem]{Lemma}
\newtheorem{proposition}[theorem]{Proposition}
\newtheorem{problem}{Problem}[theorem]
\newtheorem{conjecture}{Conjecture}[section]
\newtheorem{claim}{Claim}[theorem]
\newtheorem*{claim*}{Claim}

\newtheorem{fact}{Fact}[theorem]
\newtheorem{assumption}{Assumption}

\theoremstyle{remark}
\newtheorem{construction}{Construction}[theorem]

\theoremstyle{definition}
\newtheorem{definition}[theorem]{Definition}
\newtheorem{axiom}{Axiom}[theorem]
\newtheorem{example}{Example}
\newtheorem*{example*}{Example}

\theoremstyle{plain}
\newtheorem{defcorollary}{Corollary}[theorem]

\newtheoremstyle{nbremark}%
    {}{}{\normalfont}{}{\raisebox{-0.5mm}{\;{\Large\lefthand}\;}\itshape}%
    {.\ }{  }{}

\theoremstyle{nbremark}
\newtheorem*{remark*}{Remark}
\newtheorem{remark}{Remark}[theorem]
\newtheorem{defremark}{Remark}[theorem]
\newtheorem{secremark}{Remark}[section]
\newtheorem{observation}{Observation}[theorem]
\newtheorem*{observation*}{Observation}
\newtheorem{demonstration}{Demonstration}[theorem]
\newtheorem*{demonstration*}{Demonstration}

%% Inline remark*
\newcommand\inlineremark[1]{\begin{remark}#1\end{remark}}
\WithSuffix\newcommand\inlineremark*[1]{\begin{remark*}#1\end{remark*}}

%% Theorem referencing
\makeatletter
\newcommand{\thref}[1]{\@splitref#1\@nil}
%\def\@splitref#1:#2\@nil{{\bfseries\small\capitalisewords{#1}~\ref{#1:#2}}}
\def\@splitref#1:#2\@nil{{{#1}~\ref{#1:#2}}}
\makeatother

%% Figure referencing.
\newcommand{\figref}[1]{figure~\ref{fig:#1}}

%% Paragraph formattting.
\setlength{\parindent}{5ex}
\setlength{\parskip}{1ex plus 2pt}


%% Page geometry
\geometry{
    a4paper,
    textwidth=135mm,
    top=35mm,
}


%% Section, subsection and section format
\titleformat{\section}[block]%
	{\large}%
	{\rlap{{\large\S}\,\thesection.}}%
	{0pt}%
	{\scshape\hspace*{.05\textwidth}\begin{minipage}[t]{.9\textwidth}\centering}%
	[\end{minipage}\vspace{5pt}]

\titleformat{\subsection}[runin]%
	{}%
	{\S\,\thesubsection.}%
	{2ex}%
	{\bfseries}[.]

\titleformat{\subsubsection}[runin]%
	{}%
	{\S\,\thesubsubsection.}%
	{2ex}%
	{\bfseries}[.]

%% Useful commands
\makeatletter
\newcommand\cdotfill{%
    \leavevmode\cleaders\hb@xt@.44em{\hss$\cdot$\hss}\hfill\kern\z@
}
\makeatother

%% Itemize and enumeratge styling
%\usepackage[shortlabels]{enumitem}
\setlength{\labelsep}{1em}
\setlist{wide=3pt,leftmargin=*,align=right}
\setlist[itemize]{label=\scalebox{0.7}{\textbullet}}

\DeclareFontFamily{OT1}{pzc}{}
\DeclareFontShape{OT1}{pzc}{m}{it}%
{<-> s * [1.15] pzcmi7t}{}
\DeclareMathAlphabet{\mathpzc}{OT1}{pzc}{m}{it}

\mathtoolsset{showonlyrefs=true,centercolon,mathic=true}
\newtagform{noparen}{---\ (}{)}
\usetagform{noparen}
\renewcommand{\eqref}[1]{{\rm(\refeq{#1})}}
\renewcommand{\theequation}{\arabic{section}.\Roman{equation}}

% Inner product
\DeclarePairedDelimiterX{\inp}[2]{\langle}{\rangle}{#1, #2}
% Floor and ceil
\DeclarePairedDelimiter\ceil{\lceil}{\rceil}
\DeclarePairedDelimiter\floor{\lfloor}{\rfloor}

\newcommand\bfit[1]{\textbf{\textit{#1}}}

\newcommand\avg[1]{\left\langle{#1}\right\rangle}
\mathchardef\Re="023C
\mathchardef\Im="023D
\let\oldRe\Re
\let\oldIm\Im
\renewcommand\Re[1]{\oldRe\mathfrak{e}\left\{#1\right\}}
\renewcommand\Im[1]{\oldIm\mathfrak{m}\left\{#1\right\}}
\newcommand\C{\mathbb{C}}
\newcommand\R{\mathbb{R}}
\newcommand\Q{\mathbb{Q}}
\newcommand\N{\mathbb{N}}
\newcommand\Z{\mathbb{Z}}
\newcommand\lhs{\text{L.H.S.}}
\newcommand\rhs{\text{R.H.S.}}
\newcommand\defeq{\coloneqq}
\newcommand\ident{\equiv}
\newcommand{\dif}[1]{\mathop{{\rm d}#1}}
\newcommand{\del}{\spoon{\nabla}}
\renewcommand{\grad}[1]{{\del}{#1}}
\renewcommand{\div}[1]{{\del}\cdot{#1}}
\renewcommand{\curl}[1]{{\del}\cross{#1}}
\renewcommand{\laplacian}[1]{{\nabla}^2{#1}}
\newcommand\et{{\textit{\&}\;}}
\newcommand\etc{\mbox{\textit{\&\hspace{-0.7pt}c}.\@\xspace}}
\newcommand\ie{\mbox{\textit{i.\hspace{-1.2pt}e}.\@\xspace}}
\newcommand\eg{\textit{e.\hspace{-1pt}g}.\@\xspace}
\newcommand{\mf}{\mathfrak}
\newcommand{\mbf}{\mathbf}
\newcommand{\bs}{\textbackslash}

%% Questions styling.
\SetLabelAlign{parright}{\parbox[t]{\labelwidth}{\raggedleft#1}}
\newlist{questions}{enumerate}{3}
\setlist[questions]{itemsep=5mm,listparindent=\parindent}
\setlist[questions,1]{align=left,label={\arabic*.}}
\setlist[questions,2]{align=left,labelwidth=4ex,label={(\alph*)}}
\setlist[questions,3]{align=left,label={(\roman*)},labelwidth=7mm}
\newcommand\question[2]{\item #1\hfill[#2]}

\newcommand{\markshere}[1]{%
    \ifmmode\eqno\else\hspace*{\fill}\fi
    \textrm{[#1]}}

\newcommand\qitem[2][\relax]{\item #2{%
    \phantom{1pt}%
    \ifx\relax#1 \ \else \markshere{#1}\fi}%
	\par}

\makeatletter
\newcommand{\skipitems}[1]{%
  \addtocounter{\@enumctr}{#1}}
\makeatother

%\setlength{\jot}{10pt}

\let\abs\undefined
\let\norm\undefined
\DeclarePairedDelimiter\abs{\lvert}{\rvert}%
\DeclarePairedDelimiter\norm{\lVert}{\rVert}%

\makeatletter
\let\oldabs\abs
\def\abs{\@ifstar{\oldabs}{\oldabs*}}

\let\oldnorm\norm
\def\norm{\@ifstar{\oldnorm}{\oldnorm*}}
\makeatother

\newcommand{\infdiv}{D\infdivx}
\DeclarePairedDelimiter{\enorm}{\lVert}{\rVert}

\NewDocumentCommand{\evalat}{sO{\big}mm}{%
  \IfBooleanTF{#1}
   {\mleft. #3 \mright|_{#4}}
   {#3#2|_{#4}}%
}

\newcommand{\seq}[1]{({#1}_n)}
\newcommand{\dseq}[1]{\seq{\vec{#1}}}

\newcommand\m{\:\textrm{m}}
\newcommand\M{\:\Big[\textrm{m}\Big]}
\newcommand\mm{\:\textrm{mm}}
\newcommand\MM{\:\Big[\textrm{mm}\Big]}
\newcommand\un{\underline}
\newcommand\s{\:\textrm{s}}
\newcommand\bS{\:\Big[\textrm{S}\Big]}
\newcommand\ms{\:\frac{\textrm{m}}{\textrm{s}}}
\newcommand\MS{\:\Big[\frac{\textrm{m}}{\textrm{s}}\Big]}
\newcommand\mss{\:\frac{\textrm{m}}{\textrm{s}^2}}
\newcommand\MSS{\:\Big[\frac{\textrm{m}}{\textrm{s}^2}\Big]}

\makeatletter
\newcommand*\MY@leftharpoonupfill@{
    \arrowfill@\leftharpoonup\relbar\relbar
}
\newcommand*\MY@rightharpoonupfill@{
    \arrowfill@\relbar\relbar\rightharpoonup
}
\newcommand*\overleftharpoon{
    \mathpalette{\overarrow@\MY@leftharpoonupfill@}
}
\newcommand*\overrightharpoon{
    \mathpalette{\overarrow@\MY@rightharpoonupfill@}
}

\newcommand*\@dblsty@mathpalette[2]{
    \mathchoice
        {#1\displaystyle       \scriptstyle       {#2}}
        {#1\textstyle          \scriptstyle       {#2}}
        {#1\scriptstyle        \scriptscriptstyle {#2}}
        {#1\scriptscriptstyle  \scriptscriptstyle {#2}}
}
\newcommand*\@dblsty@overarrow@[4]{
    \vbox{\ialign{##\crcr
        #1#3\crcr
        \noalign{\nointerlineskip}
        \(\m@th\hfil #2#4\hfil\)\crcr
    }}
}
\newcommand*\smalloverleftharpoon{%
    \@dblsty@mathpalette{\@dblsty@overarrow@\MY@leftharpoonupfill@}%
}
\newcommand*\smalloverrightharpoon{%
    \@dblsty@mathpalette{\@dblsty@overarrow@\MY@rightharpoonupfill@}%
}
\makeatother
\newcommand{\poon}{\overrightharpoon}
\newcommand{\spoon}{\smalloverrightharpoon}
\newcommand{\overbar}[1]{\mkern
1.5mu\overline{\mkern-1.5mu#1\mkern-1.5mu}\mkern 1.5mu}
\newcommand{\fn}[1]{\mathop{\rm #1}\nolimits\!}
\DeclareMathOperator{\interior}{int}
\DeclareMathOperator{\sgn}{sgn}
\DeclareMathOperator{\id}{id}
\DeclareMathOperator{\ord}{ord}
\DeclareMathOperator{\supp}{supp}
\DeclareMathOperator{\dom}{dom}
\DeclareMathOperator{\ran}{ran}
\DeclareMathOperator{\cov}{cov}
\DeclareMathOperator{\variance}{Var} \newcommand\Var\variance
\DeclareMathOperator{\U}{U}
\DeclareMathOperator{\Exp}{Exp}
\DeclareMathOperator{\Gam}{Gamma}
\DeclareMathOperator{\Bet}{Beta}
\DeclareMathOperator{\Norm}{N}
\DeclareMathOperator{\NegBin}{NegBin}
\newcommand\cum\cdf
\newcommand\pmass\pms
\newcommand\pms[1]{\mathop{p_{\sub #1}}\nolimits}
\newcommand\cdf[1]{\mathop{F_{\sub #1}}\nolimits}
\newcommand\pdf[1]{\mathop{f_{\sub #1}}\nolimits}
\DeclareMathOperator{\Po}{Po}
\newcommand\HOM{({\footnotesize\bfseries\scshape{Hom}})}
\DeclareMathOperator{\Hom}{Hom}
\DeclareMathOperator{\Aut}{Aut}
\newcommand\Power{\mathop{\mathcal{P}}\nolimits}
\DeclareMathOperator{\Ker}{Ker}
\newcommand\inv{^{-1}}
\newcommand\eps{\varepsilon}
\newcommand\tht{\vartheta}
\newcommand\ph{\varphi}
\newcommand\BW{Bolzano-\!Weierstra{\ss}}
\newcommand\coord{co\"ordinate}
\newcommand{\GL}[2][1]{G\!L_{#1}\!\left(#2\right)}
\newcommand{\set}[1]{\left\{\,#1\,\right\}}
\newcommand{\excl}[1]{\setminus\set{#1}}
\newcommand{\st}{\, : \,}
\newcommand{\Lim}[1]{\raisebox{0.5ex}{\scalebox{0.8}{$\displaystyle
\lim_{#1}\;$}}}

%% Big version of math operators
\usepackage{scalerel}
\newcommand{\bigop}[1]{\mathop{\scalerel*{#1}{\sum}}\limits}

\newcommand{\super}{\textsuperscript}
\newcommand{\degree}{{\si{\degree}}}
\renewcommand{\deg}{\mathop{\rm deg}\nolimits}
\newcommand{\coef}{\mathop{\rm coef}\nolimits}
\newcommand{\numero}{N\super{\underline{o}}}
\newcommand{\ihat}{\hat{{\imath}}}
\newcommand{\jhat}{\hat{{\jmath}}}
\newcommand{\khat}{\hat{k}}

\newcommand{\smallerrel}[1]{\mathrel{\mathpalette\smallerrelaux{#1}}}
\newcommand{\smallerrelaux}[2]{\raisebox{.1ex}{\scalebox{.75}{$#1#2$}}}

\newcommand{\tinyrel}[1]{\mathrel{\mathpalette\tinyrelaux{#1}}}
\newcommand{\tinyrelaux}[2]{\raisebox{.1ex}{\scalebox{.55}{$#1#2$}}}

\newcommand{\shrink}{\smallerrel}
\newcommand{\Shrink}{\tinyrel}
\NewDocumentCommand{\pad}{O{1ex}O{1ex}m}{\hspace*{#1}#3\hspace*{#2}}
\newcommand{\lpad}[2][2ex]{\pad[#1][0pt]{#2}}
\newcommand{\rpad}[2][2ex]{\pad[0.01pt][#1]{#2}}
%% Defining operators
\let\Implies\implies
\let\Impliedby\impliedby
\let\Iff\iff
\renewcommand\implies{\;\mathop{\Rightarrow}\;}
\renewcommand\impliedby{\;\mathop{\Leftarrow}\;}
\renewcommand\iff{\;\mathop{\Leftrightarrow}\;}
\newcommand{\proofleft}{{($\Longleftarrow$)\quad}}
\newcommand{\proofright}{{($\Longrightarrow$)\quad}}
\newcommand\To\longrightarrow
\newcommand\Mapsto\longmapsto
\newcommand\mequiv\Leftrightarrow  %% Material equivalence.
\newcommand{\comp}{\mathop{\shrink{\circ}}}
\newcommand{\iron}{\mathop{\Shrink{\maltese}}}
%\newcommand{\compl}[1]{{#1}^\complement}
\newcommand{\compl}[1]{\overline{#1}}
\newcommand\given{\mathop{|}}
\newcommand\mbar{\;\middle|\;}
\newcommand\forany\forall
\newcommand\any\forall
\newcommand\forsome\exists
\newcommand\some\exists
\newcommand\exactlyone{\exists!}
\newcommand\onlyone{\exists!}
\newcommand\union{\cup}
\newcommand\inter{\cap}
\newcommand\by\cross

\newcommand{\sub}{\mathchoice{}{}{\scriptscriptstyle}{}}
\newcommand{\ssub}[1]{_{\sub{#1}}}

\newcommand{\then}{\Rightarrow\ }
\newcommand{\btw}[1]{\noalign{\centering\parbox{8cm}{#1}}}
\newcommand{\asside}[1]{\qquad\text{\small #1}}

\renewcommand\pmat[1]{\begin{pmatrix}#1\end{pmatrix}}
\newcommand{\mat}[2][b]{\begin{#1matrix}#2\end{#1matrix}}
\newcommand\bmat{\mat}
\newcommand\detmat[1]{\begin{vmatrix}#1\end{vmatrix}}
\newenvironment{amatrix}[2]{%
	\left[\begin{array}{*{#1}{c}|*{#2}{c}}
}{%
  \end{array}\right]
}

\usepackage{fourier-orns}

%\makeatletter
%\newcommand{\pgfplotsdrawaxis}{\pgfplots@draw@axis}
%\makeatother
%\pgfplotsset{only axis on top/.style={axis on top=false, after end
axis/.code={
%             \pgfplotsset{axis line style=opaque, ticklabel style=opaque, tick
style=opaque,
%                          grid=none}\pgfplotsdrawaxis}}}
%
\newcommand{\drawge}{-- (rel axis cs:1,0) -- (rel axis cs:1,1) -- (rel axis
cs:0,1) \closedcycle}
\newcommand{\drawle}{-- (rel axis cs:1,1) -- (rel axis cs:1,0) -- (rel axis
cs:0,0) \closedcycle}


\newcommand{\NB}{\raisebox{-1mm}{\,{\Large \lefthand}\ \,}}
%\newcommand{\NB}{\,{\bfseries N\hspace{-1.6mm}B}\ \,}

\newcommand{\spacer}[1]{%
    \vspace*{\fill}
    \hspace*{-\leftmargin}{\hfil\itshape#1\par}
    \vspace*{\fill}}

\renewcommand{\arraystretch}{1.07}

%% Tightly spaced math lines.
\newenvironment{tight}[1][0ex]%
	{\spreadlines{#1}\align}%
	{\endalign\endspreadlines}

%% xy-pic diagrams
\usepackage{environ}
\NewEnviron{diag}[1][1.1]{%
	\[\scalebox{#1}{\xymatrix{\BODY}}\]%
}

\tikzset{
  compass/.pic = {
    \foreach[count=\i,evaluate={\m=div(\i-1,4);\a=90*\i-45*(\m+1)}] \d in
{NE,NW,SW,SE,E,N,W,S}{
      \filldraw[pic actions,rotate=\a,scale=.7+.3*\m] (0,0) -- (45:1)--(0:3)
node[transform shape,rotate=-90,above,yshift=3pt]{\bf\d};
      \filldraw[pic actions,fill=white,rotate=\a,scale=.7+.3*\m] (0,0) --
(-45:1)--(0:3)--cycle;
    };
  }
}

%% tikz drawing permutations.
\newcommand{\permut}[3][1]{%
	\def\gap{1.2pt}%  % Larger gap with arrow `Bar-To` is also pretty.
	\def\dot{1.2pt}%
	\def\dotstyle{teal!80}%
	\tikz[baseline={#1 mm * 5 - 2}]{%
		% Draw top row numbers, (domain).
		\foreach \elem [count=\i] in {#2} {%
			\draw (\i, #1) node[anchor=south]{\(\elem\)};%
		}%
		% Draw arrows to  next row of numbers.
		\foreach \level in {#1,...,1} {%
			\foreach \next [count=\i] in {#3} {%
				% Find index of this element (\next), in {#2}.
				\foreach \elem [count=\j] in {#2} {%
					\ifthenelse{\equal{\next}{\elem}}%
						{\draw[thick,->,shorten >=\gap,shorten <=\gap]%
							(\i, \level) -- (\j, {\level - 1});}%
						{}%
				}%
				\draw[fill=\dotstyle] (\i, \level) circle(\dot);%
			}%
		}%
		% Draw bottom row numbers and circles, (range).
		\foreach \elem [count=\i] in {#2} {%
			\draw[fill=\dotstyle] (\i, 0) circle(\dot)%
				node[anchor=north]{\(\elem\)};%
		}%
	}%
}

%% Polygon for TikZ
\NewDocumentCommand{\polygon}{O{}O{}mO{-90}}{
  \pgfmathsetmacro{\angle}{360/#3}
  \pgfmathsetmacro{\startangle}{#4 + \angle/2}
  \pgfmathsetmacro{\y}{cos(\angle/2)}
  \begin{scope}[#2]
    \foreach \i in {1,2,...,#3} {
      \pgfmathsetmacro{\x}{\startangle + \angle*\i}
	  \draw[#1] (\x:1) -- (\x + \angle:1);
    }
  \end{scope}
}

%%% TikZ Example %%%

% \begin{tikzpicture}
%     \begin{axis}[
%         axis lines = middle,
%         xlabel = $x$,
%         ylabel = $f(x)$,
%         ylabel near ticks,
%         grid = major,
%         xtick = {4.85, 3, -1, -1.85},
%         ytick = {5, -27},
%         ymax = 10,
%         ymin = -30,
%         xmax = 6,
%         width=13cm,
%         height=9cm
%     ]
%     %\addplot[domain=0:370]{}
%     \addplot [
%         domain=-3:8,
%         samples=200,
%         color=red,
%     ]
%     {x^3 - 3*x^2 - 9*x};
%     \addlegendentry{$x^3 - 3x^2 - 9x$}

%     \end{axis}
% \end{tikzpicture}
