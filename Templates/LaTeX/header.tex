\usepackage[utf8]{inputenc}
\usepackage[UKenglish]{babel}
\usepackage[UKenglish]{datetime}
\usepackage[T1]{fontenc}
\usepackage{geometry}
\usepackage{xspace}
\usepackage{varwidth}
\usepackage{svg}
\usepackage{tikz}
\usetikzlibrary{calc,math,arrows,arrows.meta,decorations.pathreplacing}
\usepackage{pgfplots}
\usepackage{pgffor}
\usepackage{pgfmath}
\usepackage[super]{nth}
\usepackage{romannum}
\AtBeginDocument{\pagenumbering{arabic}}
\usepackage{float}
\usepackage{caption}
\usepackage{makecell}
\usepackage{tabularx}
\usepackage{multirow}
\renewcommand\tabularxcolumn[1]{m{#1}}
\usepackage{array}
\usepackage{amsmath,mleftright,amsthm,amsfonts,amssymb,amscd,nccmath}
\usepackage{centernot}
\usepackage{relsize}
\usepackage{physics}
\usepackage{polynom}
\usepackage{xparse}
\usepackage{fancyhdr}
\usepackage{titlesec}
\usepackage{accents}
\usepackage{mathtools}
\usepackage[scaled=1.15]{urwchancal}
\usepackage{bbding}
\usepackage{xfrac}
\usepackage{etoolbox}
\usepackage[artemisia]{textgreek}
\usepackage{siunitx}
\usepackage{parskip}
\usepackage{multicol}
\usepackage{enumerate}
\usepackage{enumitem}
\usepackage{mathtools}
\usepackage{mathrsfs}
%\usepackage{lmodern}
\usepackage{slantsc}
\usepackage{bold-extra}
\usepackage{mfirstuc}
\usepackage{suffix}


%% Cal fonts
\DeclareMathAlphabet{\pazocal}{OMS}{zplm}{m}{n}
\newcommand\call[1]{\pazocal{#1}}
\newcommand\calll[1]{\mathcal{#1}}
\newcommand\callll[1]{\mathscr{#1}}

%% Nice empty-set.
\newcommand\oldemptyset\emptyset
\renewcommand\emptyset{\mathlarger{\mathlarger\varnothing}}


%% Header layout.
\fancyhf{}
\headheight 14pt
\fancyhead[RO]{\papertitle}
\fancyhead[LE]{\paperauthor}
\fancyhead[RE,LO]{\thepage}
\pagestyle{fancy}

\makeatletter
\let\papertitle\@title
\let\paperauthor\@author
\makeatother


%% Theorems, definitions, remarks, lemmas, corrolaries, &c.
\theoremstyle{plain}
\newtheorem{theorem}{Theorem}[section]
\newtheorem*{theorem*}{Theorem}
\newtheorem{corollary}{Corollary}[theorem]
\newtheorem{lemma}{Lemma}[theorem]
\newtheorem{lemmaalone}{Lemma}[section]
\newtheorem{proposition}{Proposition}[theorem]
\newtheorem{problem}{Problem}[theorem]
\newtheorem{conjecture}{Conjecture}[theorem]
\newtheorem{claim}{Claim}[theorem]
\newtheorem*{claim*}{Claim}

\newtheorem{fact}{Fact}[theorem]
\newtheorem{assumption}{Assumption}

\theoremstyle{remark}
\newtheorem{construction}{Construction}[theorem]
\newtheorem{observation}{Observation}[theorem]

\theoremstyle{definition}
\newtheorem{definition}{Definition}[section]
\newtheorem{axiom}{Axiom}[definition]
\newtheorem{example}{Example}

\newtheoremstyle{nbremark}%
    {}{}{\normalfont}{}{\raisebox{-0.5mm}{\;{\Large\lefthand}\;}\itshape}%
    {.\ }{  }{}
\theoremstyle{nbremark}

\newtheorem*{remark*}{Remark}
\newtheorem{remark}{Remark}[theorem]
\newtheorem{defremark}{Remark}[definition]
\newtheorem{secremark}{Remark}[section]


%% Inline remark*
\newcommand\inlineremark[1]{\begin{remark}#1\end{remark}}
\WithSuffix\newcommand\inlineremark*[1]{\begin{remark*}#1\end{remark*}}

%% Theorem referencing
\makeatletter
\newcommand{\thref}[1]{\@splitref#1\@nil}
%\def\@splitref#1:#2\@nil{{\bfseries\small\capitalisewords{#1}~\ref{#1:#2}}}
\def\@splitref#1:#2\@nil{{{#1}~\ref{#1:#2}}}
\makeatother

%% Paragraph formattting.
\setlength{\parindent}{5ex}
\setlength{\parskip}{1ex}


%% Page geometry
\geometry{
    a4paper,
    textwidth=135mm,
    top=35mm,
}


%% Section, subsection and section format
\titleformat{\section}[block]%
	{\large}%
	{\rlap{{\large\S}\,\thesection.}}%
	{0pt}%
	{\scshape\hspace*{.05\textwidth}\begin{minipage}[t]{.9\textwidth}\centering}%
	[\end{minipage}\vspace{5pt}]

\titleformat{\subsection}[runin]%
	{}%
	{\S\,\thesubsection.}%
	{2ex}%
	{\bfseries}[.]

\titleformat{\subsubsection}[runin]%
	{}%
	{\S\,\thesubsubsection.}%
	{2ex}%
	{\bfseries}[.]

%\usepackage[shortlabels]{enumitem}
\setlength{\labelsep}{1em}
\setlist{wide=0pt,leftmargin=*}

\DeclareFontFamily{OT1}{pzc}{}
\DeclareFontShape{OT1}{pzc}{m}{it}%
{<-> s * [1.15] pzcmi7t}{}
\DeclareMathAlphabet{\mathpzc}{OT1}{pzc}{m}{it}

\mathtoolsset{showonlyrefs}
\newtagform{noparen}{---\ (}{)}
\usetagform{noparen}
\renewcommand{\eqref}[1]{(\refeq{#1})}
\renewcommand{\theequation}{\arabic{section}.\Roman{equation}}

% Inner product
\DeclarePairedDelimiterX{\inp}[2]{\langle}{\rangle}{#1, #2}
% Floor and ceil
\DeclarePairedDelimiter\ceil{\lceil}{\rceil}
\DeclarePairedDelimiter\floor{\lfloor}{\rfloor}

\newcommand\bfit[1]{\textbf{\textit{#1}}}

\newcommand\avg[1]{\left\langle{#1}\right\rangle}
\mathchardef\Re="023C
\mathchardef\Im="023D
\let\oldRe\Re
\let\oldIm\Im
\renewcommand\Re[1]{\oldRe\mathfrak{e}\left\{#1\right\}}
\renewcommand\Im[1]{\oldIm\mathfrak{m}\left\{#1\right\}}
\newcommand\C{\mathbb{C}}
\newcommand\R{\mathbb{R}}
\newcommand\Q{\mathbb{Q}}
\newcommand\N{\mathbb{N}}
\newcommand\Z{\mathbb{Z}}
\newcommand\lhs{\text{L.H.S.}}
\newcommand\rhs{\text{R.H.S.}}
\newcommand\defeq{\coloneqq}
\newcommand*{\dif}[1]{\mathop{{\rm d}#1}}
\newcommand\et{{\;\textit{\&}\:}}
\newcommand\etc{\textit{\&\hspace{-0.7pt}c}.\@\xspace}
\newcommand\ie{\textit{i.\hspace{-1.2pt}e}.\@\xspace}
\newcommand\eg{\textit{e.\hspace{-1pt}g}.\@\xspace}
\newcommand*{\mf}{\mathfrak}
\newcommand{\bs}{\textbackslash}
\newlist{questions}{enumerate}{3}
\setlist[questions]{itemsep=5mm,listparindent=\parindent}
\setlist[questions,1]{align=left,label={\arabic*.}}
\setlist[questions,2]{align=left,labelwidth=4ex,label={(\alph*)}}
\setlist[questions,3]{align=left,label={(\roman*)},labelwidth=7mm}
\newcommand\question[2]{\item #1\hfill[#2]}

\newcommand{\markshere}[1]{%
    \ifmmode\eqno\else\hspace*{\fill}\fi
    \textrm{[#1]}}

\newcommand\qitem[2][\relax]{\item #2{%
    \phantom{1pt}%
    \ifx\relax#1 \ \else \markshere{#1}\fi}}

\makeatletter
\newcommand{\skipitems}[1]{%
  \addtocounter{\@enumctr}{#1}}
\makeatother

\setlength{\jot}{10pt}

\let\abs\undefined
\let\norm\undefined
\DeclarePairedDelimiter\abs{\lvert}{\rvert}%
\DeclarePairedDelimiter\norm{\lVert}{\rVert}%

\makeatletter
\let\oldabs\abs
\def\abs{\@ifstar{\oldabs}{\oldabs*}}

\let\oldnorm\norm
\def\norm{\@ifstar{\oldnorm}{\oldnorm*}}
\makeatother

\newcommand{\infdiv}{D\infdivx}
\DeclarePairedDelimiter{\enorm}{\lVert}{\rVert}

\NewDocumentCommand{\evalat}{sO{\big}mm}{%
  \IfBooleanTF{#1}
   {\mleft. #3 \mright|_{#4}}
   {#3#2|_{#4}}%
}

\newcommand\m{\:\textrm{m}}
\newcommand\M{\:\Big[\textrm{m}\Big]}
\newcommand\mm{\:\textrm{mm}}
\newcommand\MM{\:\Big[\textrm{mm}\Big]}
\newcommand\un{\underline}
\newcommand\s{\:\textrm{s}}
\newcommand\bS{\:\Big[\textrm{S}\Big]}
\newcommand\ms{\:\frac{\textrm{m}}{\textrm{s}}}
\newcommand\MS{\:\Big[\frac{\textrm{m}}{\textrm{s}}\Big]}
\newcommand\mss{\:\frac{\textrm{m}}{\textrm{s}^2}}
\newcommand\MSS{\:\Big[\frac{\textrm{m}}{\textrm{s}^2}\Big]}

\makeatletter
\newcommand*\MY@leftharpoonupfill@{
    \arrowfill@\leftharpoonup\relbar\relbar
}
\newcommand*\MY@rightharpoonupfill@{
    \arrowfill@\relbar\relbar\rightharpoonup
}
\newcommand*\overleftharpoon{
    \mathpalette{\overarrow@\MY@leftharpoonupfill@}
}
\newcommand*\overrightharpoon{
    \mathpalette{\overarrow@\MY@rightharpoonupfill@}
}

\newcommand*\@dblsty@mathpalette[2]{
    \mathchoice
        {#1\displaystyle       \scriptstyle       {#2}}
        {#1\textstyle          \scriptstyle       {#2}}
        {#1\scriptstyle        \scriptscriptstyle {#2}}
        {#1\scriptscriptstyle  \scriptscriptstyle {#2}}
}
\newcommand*\@dblsty@overarrow@[4]{
    \vbox{\ialign{##\crcr
        #1#3\crcr
        \noalign{\nointerlineskip}
        \(\m@th\hfil #2#4\hfil\)\crcr
    }}
}
\newcommand*\smalloverleftharpoon{%
    \@dblsty@mathpalette{\@dblsty@overarrow@\MY@leftharpoonupfill@}%
}
\newcommand*\smalloverrightharpoon{%
    \@dblsty@mathpalette{\@dblsty@overarrow@\MY@rightharpoonupfill@}%
}
\makeatother
\newcommand{\poon}{\overrightharpoon}
\newcommand{\spoon}{\smalloverrightharpoon}
\newcommand\dom{\mathop{\rm dom}\nolimits}
\newcommand\ran{\mathop{\rm ran}\nolimits}
\newcommand\variance{\mathop{\rm var}\nolimits}
\newcommand\pmass[1]{\mathop{p_{\sub #1}}\nolimits}
\newcommand\cum[1]{\mathop{F_{\sub #1}}\nolimits}
\newcommand\Po{\mathop{\rm Po}\nolimits}
\newcommand{\set}[1]{\left\{\,#1\,\right\}}
\newcommand{\st}{\: : \:}
\newcommand{\Lim}[1]{\raisebox{0.5ex}{\scalebox{0.8}{$\displaystyle \lim_{#1}\;$}}}

\newcommand{\super}{\textsuperscript}
\renewcommand{\deg}{{\si{\degree}}}
\newcommand{\numero}{N\super{\underline{o}}}
\newcommand{\ihat}{\hat{{\imath}}}
\newcommand{\jhat}{\hat{{\jmath}}}
\newcommand{\khat}{\hat{k}}

\newcommand{\smallerrel}[1]{\mathrel{\mathpalette\smallerrelaux{#1}}}
\newcommand{\smallerrelaux}[2]{\raisebox{.1ex}{\scalebox{.75}{$#1#2$}}}

\newcommand{\shrink}{\smallerrel}
%% Defining operators
\let\Implies\implies
\let\Impliedby\impliedby
\let\Iff\iff
\renewcommand\implies{\;\mathop{\Rightarrow}\;}
\renewcommand\impliedby{\;\mathop{\Leftarrow}\;}
\renewcommand\iff{\;\mathop{\Leftrightarrow}\;}
\newcommand\mequiv\Leftrightarrow  %% Material equivalence.
\newcommand{\comp}{\mathbin{\shrink{\circ}}}
%\newcommand{\compl}[1]{{#1}^\complement}
\newcommand{\compl}[1]{\overline{#1}}
\newcommand\given{\mathop{|}}
\newcommand\forany\forall
\newcommand\forsome\exists
\newcommand\exactlyone{\exists!}
\newcommand\onlyone{\exists!}

\newcommand{\sub}{\mathchoice{}{}{\scriptscriptstyle}{}}

\newcommand{\then}{\Rightarrow\ }
\newcommand{\btw}[1]{\noalign{\centering\parbox{8cm}{#1}}}
\newcommand{\asside}[1]{\qquad\text{\small #1}}

\newcommand\mat[1]{\begin{bmatrix}#1\end{bmatrix}}
\newcommand\detmat[1]{\begin{vmatrix}#1\end{vmatrix}}

\usepackage{fourier-orns}

\usetikzlibrary{patterns}

\makeatletter
\newcommand{\pgfplotsdrawaxis}{\pgfplots@draw@axis}
\makeatother
\pgfplotsset{only axis on top/.style={axis on top=false, after end axis/.code={
             \pgfplotsset{axis line style=opaque, ticklabel style=opaque, tick style=opaque,
                          grid=none}\pgfplotsdrawaxis}}}

\newcommand{\drawge}{-- (rel axis cs:1,0) -- (rel axis cs:1,1) -- (rel axis cs:0,1) \closedcycle}
\newcommand{\drawle}{-- (rel axis cs:1,1) -- (rel axis cs:1,0) -- (rel axis cs:0,0) \closedcycle}


\newcommand{\NB}{\raisebox{-1mm}{\,{\Large \lefthand}\ \,}}
%\newcommand{\NB}{\,{\bfseries N\hspace{-1.6mm}B}\ \,}

\newcommand{\spacer}[1]{%
    \vspace*{\fill}
    \hspace*{-\leftmargin}{\hfil\itshape#1\par}
    \vspace*{\fill}}

\renewcommand{\arraystretch}{1.2}

%%% TikZ Example %%%

% \begin{tikzpicture}
%     \begin{axis}[
%         axis lines = middle,
%         xlabel = $x$,
%         ylabel = $f(x)$,
%         ylabel near ticks,
%         grid = major,
%         xtick = {4.85, 3, -1, -1.85},
%         ytick = {5, -27},
%         ymax = 10,
%         ymin = -30,
%         xmax = 6,
%         width=13cm,
%         height=9cm
%     ]
%     %\addplot[domain=0:370]{}
%     \addplot [
%         domain=-3:8,
%         samples=200,
%         color=red,
%     ]
%     {x^3 - 3*x^2 - 9*x};
%     \addlegendentry{$x^3 - 3x^2 - 9x$}

%     \end{axis}
% \end{tikzpicture}
